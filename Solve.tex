\section{Solving the PageRank Equation} \label{sec:solve}
In order to solve the PageRank equation we either need to solve the eigenvector problem for $\boldsymbol{\pi}^T$, \(\boldsymbol{\pi}^T = \boldsymbol{\pi}^T\textbf{G}\), i.e. we need to find a normalized dominant left-hand eigenvector of \textbf{G} which corresponds to the dominant eigenvalue of $\lambda_1 = 1$. We are also able to solve the equation by using a linear system \(\boldsymbol{\pi}^T(\textbf{I}-\textbf{G})=\textbf{0}^T\), where we want to find a normalized left-hand vector of \textbf{I}-\textbf{G}, \cite{langville}. We will discuss the power method in more detail in this report, as this is the original method proposed to solve the PageRank equation \cite{langville}.

\subsection{Power Method} \label{sec:power}
PERRON-FROBENIUS \cite{meyer2000matrix}  \cite{gallager1992discrete} 
\textcolor{red}{\textbf{EXPLAIN MATHS MORE HERE}} Perron-Frobenius theorem that a positive, stochastic, irreducible matrix is guaranteed to have a positive eigenvector \cite{thorson2004modeling}.

Using Matlab code as given in Appendix \ref{app:code}, 

\[\boldsymbol\pi = \left(
\begin{array}{c}
0.040 \\
0.051 \\
0.062 \\
0.066 \\
0.258 \\
0.051 \\
0.199 \\
0.274
\end{array}
\right)\]

\begin{table}[H] \caption{Ranking of web pages after two iterations and after the power method}
 \centering
 \begin{tabular} {c| c c} 
 Node & Rank at Iter. 2 & PR \\ [0.5ex] 
 \hline
 1&6&8\\
 2&6&6\\
 3&4&5\\
 4&4&4\\
 5&2&2\\
 6&6&6\\
 7&3&3\\
 8&1&1\\
 \end{tabular}
 \label{Table:PR and summ}
\end{table}

\subsection{Why the Power Method}\label{sec:why power}

The power method is useful for solving the PageRank equation as it converges to a unique vector, the PageRank vector, where the $i^{th}$ entry of $\boldsymbol\pi$ relates to the PageRank of \textit{i} \cite{ipsen2005analysis}. The main disadvantage of the power method is that it known to be very slow to converge, with the computation of the PageRank vector taking a few days to complete for the large web graph, but the power method is very simple. As we are able to simplify \textbf{G} to \textbf{H} as shown in Equation \eqref{eq:G in H}, the power method is very storage friendly, this is as for an iteration only a sparse \textbf{H}, a dangling node vector \textbf{a} and the current iterate are required to be stored \cite{langville}. Brin and Page reported that the PageRank vector converges to 2-3 degrees of accuracy within 50-100 iterations \cite{austin}. Pages with lower PR are converge very quickly when compared to pages with higher PR \cite{thorson2004modeling}.